\documentclass[a4paper, 14pt]{article}
\usepackage{amsmath, amsthm, amssymb, amsfonts}
\usepackage{fancyhdr}
\usepackage[T1]{fontenc}

\renewcommand{\contentsname}{Sadr\v zaj}

\pagestyle{fancy}
\fancyhead[r]{\large{\thepage}}
\fancyfoot{}

\newtheorem{example}{Primjer}

\begin{document}

\begin{titlepage}
\raggedright
\rule{1pt}{\textheight}
\hspace{0.05\textwidth}
\parbox[b]{0.75\textwidth}{

{\large{Prirodno-matemati\v cki fakultet\\
Univerzitet u Tuzli\\
Odsjek: \textbf{Matematika}\\
Predmet: \textbf{Uvod u programske pakete}}}\\[7\baselineskip]
{\Huge\bfseries Skupovi}\\[1\baselineskip]
{\large\textit{-seminarski rad-}}\\[5\baselineskip]
{\Large\textsc{halilovi\' c adis}}

\vspace{0.5\textheight}

{\noindent Oktobar 2019}\\[\baselineskip]
}
\end{titlepage}

\tableofcontents
\thispagestyle{empty}
\newpage

\section{Poglavlje}

\begin{center}
\huge{\textbf{Skupovi}}\\[\baselineskip]
\end{center}
\begin{Large}
Jedan od osnovnih ljudskih nagona jeste da sortira i klasificira stvari. Razmislite, naprimjer. Koliko razli\v citih skupova ste vi \v clan? Ako krenete sa nekim jednostavnim kategorijama, kao jeste li mu\v sko ili \v zensko, va\v sa starosna grupa, dr\v zava u kojoj \v zivite. Tada \v cete mo\v zda razmi\v sljati i o etni\v koj pripadnosti va\v se porodice, sociolo\v sko-ekonomskoj grupi, nacionalnosti. Ovo su samo neke od mnogobrojnih na\v cina kako bi mogli opisati sebe drugim ljudima.

Kakva je korist od ovakve kategorizacije? Kao \v sto \v cete vidjeti u nastavku ovog poglavlja, sortiraju\v ci stvari po skupovima poma\v ze vam organizirati i urediti va\v s svijet. U mogu\v cnosti ste da vladate velikim koli\v cinama informacija. Skupovi su u\v cevni alat koji poma\v ze da se odgovori na pitanje: "Koje su karakteristike grupe?"

Skupovi postavljaju temelje ostalim matemati\v kim oblastima, poput logike i apstraktne algebre. Ustvari, knjiga \textit{El\' ements de Math\' ematique}, napisana od strane grupe francuskog matemati\v cara pod pseudonimom "Nicolas Bourbaki", ka\v ze: "U dana\v snjici je mogu\' ce, logi\v cki govore\' ci: "U dana\v snjici je mogu\v ce, logi\v cki re\v ceno, "

\vfill
\noindent\fbox{\parbox{\textwidth}{Skupovi su osnovni alat za u\v cenje pa \v cak i za malu djecu. Kao bebe, oni nau\v ce razlikovati "mene" od "mama" i "tata". Kao mali\v sani nau\v ce razlikovati i kategorisati objekte kao \v clanove skupa po veli\v cini, boji, ili obliku. TV emisija "Ulice sezama" u\v ci djecu da prave skupove kroz igru. "Jedna od tih stvari je razli\v cita od ostalih."}}
\end{Large}

\newpage
\subsection{Koncepti skupa}

Svakodnevno sre\' cemo skupove u na\v sim \v zivotima u raznim situacijama. Skup je kolekcija objekata, koji se nazivaju \textbf{elementi} ili \textbf{\v clanovi} skupa. Naprimjer, Sjedinjena Ameri\v ka dr\v zava je kolekcija ili skup 50 manjih dr\v zava. 50 pojedinih dr\v zava su \v clanovi ili elementi jednog skupa koji se naziva Sjedinjene Ameri\v cke dr\v zave.
Skup je \textbf{dobro definisan} ako se njegov sadr\v zaj mo\v ze jasno odrediti. Skup trenutnih sudija koji slu\v ze SAD vrhovnom sudu je dobro definisan skup jer njihovi \v clanovi i sudije mogu biti imenovani. Skup tri najbolja auta nije dobro definisan skup jer rije\v c \textit{najbolji} razli\v citi ljudi razli\v cito tuma\v ce. U ovom tekstu, mi koristimo samo dobro definisane skupove.

Za ozna\v cavanje skupa obi\v cno se koriste sljede\v ce tri metode:
\begin{center}
opis\\
popis ili obrazac\\
notacija skupa
\end{center}
Na\v cin ozna\v cavanja skupa \textbf{opisom} prikazan je u primjeru 1.
\begin{example}

\end{example}

\end{document}