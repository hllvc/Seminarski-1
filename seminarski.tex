\documentclass[a4paper, 14pt]{article}

\usepackage{amsmath, amsthm, amssymb, amsfonts}
\usepackage{fancyhdr}
\usepackage[T1]{fontenc}
\usepackage[utf8]{inputenc}

\renewcommand{\contentsname}{Sadržaj}

\pagestyle{fancy}
\fancyhead[r]{\large{\thepage}}
\fancyfoot{}

\begin{document}
\begin{titlepage}
\raggedright
\rule{1pt}{\textheight}
\hspace{0.05\textwidth}
\parbox[b]{0.75\textwidth}{

{\large{Prirodno-matematički fakultet\\
Univerzitet u Tuzli\\
Odsjek: \textbf{Matematika}\\
Predmet: \textbf{Uvod u programske pakete}}}\\[7\baselineskip]
{\Huge\bfseries Skupovi}\\[1\baselineskip]
{\large\textit{-seminarski rad-}}\\[5\baselineskip]
{\Large\textsc{halilović adis}}

\vspace{0.5\textheight}

{\noindent Oktobar 2019}\\[\baselineskip]
}
\end{titlepage}

\tableofcontents
\thispagestyle{empty}
\newpage

\section{Skupovi}

\begin{Large}
Jedan od osnovnih ljudskih nagona jeste da sortira i klasificira stvari. Razmislite, naprimjer. Koliko različitih skupova ste vi član? Ako krenete sa nekim jednostavnim kategorijama, kao jeste li muško ili žensko, vaša starosna grupa, država u kojoj živite. Tada ćete možda razmišljati i o etničkoj pripadnosti vaše porodice, sociološko-ekonomskoj grupi, nacionalnosti. Ovo su samo neke od mnogobrojnih načina kako bi mogli opisati sebe drugim ljudima.

Kakva je korist od ovakve kategorizacije? Kao što ćete vidjeti u nastavku ovog poglavlja, sortirajući stvari po skupovima pomaže vam organizirati i urediti vaš svijet. U mogućnosti ste da vladate velikim količinama informacija. Skupovi su učevni alat koji pomaže da se odgovori na pitanje: "Koje su karakteristike grupe?"

Skupovi postavljaju temelje ostalim matematičkim oblastima, poput logike i apstraktne algebre. Ustvari, knjiga \textit{El\' ements de Math\' ematique}, napisana od strane grupe francuskog matematičara pod pseudonimom "Nicolas Bourbaki", kaže: "U današnjici je moguće, logički govoreći: "U današnjici je moguće, logički rečeno, "

\vfill
\noindent\fbox{\parbox{\textwidth}{Skupovi su osnovni alat za učenje pa čak i za malu djecu. Kao bebe, oni nauče razlikovati "mene" od "mama" i "tata". Kao mališani nauče razlikovati i kategorisati objekte kao članove skupa po veličini, boji, ili obliku. TV emisija "Ulice sezama" uči djecu da prave skupove kroz igru. "Jedna od tih stvari je različita od ostalih."}}
\end{Large}

\newpage
\subsection{Koncepti skupa}

Svakodnevno srešemo skupove u našim životima u raznim situacijama. Skup je kolekcija objekata, koji se nazivaju \textbf{elementi} ili \textbf{članovi} skupa. Naprimjer, Sjedinjena Američka država je kolekcija ili skup 50 manjih država. 50 pojedinih država su članovi ili elementi jednog skupa koji se naziva Sjedinjene Američke države.
Skup je \textbf{dobro definisan} ako se njegov sadržaj može jasno odrediti. Skup trenutnih sudija koji služe SAD vrhovnom sudu je dobro definisan skup jer njihovi članovi i sudije mogu biti imenovani. Skup tri najbolja auta nije dobro definisan skup jer riječ \textit{najbolji} različiti ljudi različito tumače. U ovom tekstu, mi koristimo samo dobro definisane skupove.

Za označavanje skupa obično se koriste sljedeće tri metode:
\begin{center}
opis\\
popis ili obrazac\\
notacija skupa
\end{center}
Način označavanja skupa \textbf{opisom} prikazan je u primjeru 1.\\

Smještanje elemenata skupa unutar velikih zagrada "\{ \}", naziva se ruster forma. Zagrade su bitan dio notacije jer ona označavaju članove skupa.\\
Naprimjer, \{1, 2, 3\} predstavlja notaciju za skup kojem pripadaju elemnti 1, 2 i 3, ali (1, 2, 3) i [1, 2, 3] nisu skup iz razloga što velike i srednje zagrade n prikazuju skup.

Skupo se inače imenuju sa velikim štampanim slovima. Naprimjer, ime koje se obično bira za skup prirodnih brojeva je \textbf{N}.\\
\begin{center}
\textbf{\textsc{PRIRODNI BROJEVI}}\\
N = \{1, 2, 3, 4, 5, ... \}
\end{center}
Tri tačke nakon 5, nazivaju se \textit{elipsa}, i označavaju da elementi skupa nastavljaju dalje na isti način. Ako se nakon elipse pojavi neki elemenat, znači da elementi idu na isti način sve do tog elementa uključujući i njega. Ovakva notacija prikazana je na primjeru ...

Simbol \textbf{$\in$}, koji se čita, elemenat je, koristi se za označavanje članova u skupu. U primjeru 3, 6 je elemenat skupa A, pišemo $6\in A$. Također možemo napisati $6\in \{6, 7\}$. Isto tako $8\notin A$, što znači da 8 nije elemenat skupa A.


\end{document}