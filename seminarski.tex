\PassOptionsToPackage{table,xcdraw,dvipsname}{xcolor}
\documentclass[a4paper,14pt,svgnames]{article}

\usepackage{amsmath, amsthm, amssymb, amsfonts}
\usepackage[shortlabels]{enumitem}
\usepackage{fancyhdr}
\usepackage[T1]{fontenc}
\usepackage[utf8]{inputenc}

\usepackage{pstricks}

\usepackage{tcolorbox}

\usepackage{titlesec,letltxmacro}
\usepackage[table,xcdraw,dvipsname]{xcolor}
\usepackage{tikz}
\usetikzlibrary{calc}
\usetikzlibrary{shapes.misc}

\renewcommand{\contentsname}{Sadržaj}

\newcounter{counter}
\newcommand{\examplecounter}{\textbf{\refstepcounter{counter}PRIMJER \thecounter}}

\usepackage{fourier-orns} 
\pagestyle{fancy}

\definecolor{mycol}{RGB}{63, 138, 135}
\fancyhf{}
\hoffset 2.2cm
\voffset -1cm
\fancyhead[L]{%
  \begin{tikzpicture}[overlay,remember picture]
      \fill [color=mycol]
        (current page.north west)
        rectangle
        ($ (current page.south west) + (5cm,0cm) $);
  \end{tikzpicture}
\begin{tikzpicture}[overlay,remember picture]
      \fill [color=white]
        (current page.north west)
        rectangle
        ($ (current page.south west) + (0.3cm,0cm) $);
  \end{tikzpicture}
}
\renewcommand{\headrulewidth}{0pt}
%\fancyhead[C]{\textsc{Skupovi} --- \textsc{Adis Halilović}}

\fancyfoot[C]{-~\thepage~-}
\renewcommand\footrule{%
\hrulefill \raisebox{-2.1pt} {\quad\decofourleft\decotwo\decofourright\quad}%
\hrulefill}
\fancyfootoffset[R,L]{-0.5\textwidth}

\newcommand\titlebar{%
    \tikz[baseline,trim left=3em,trim right=3cm] {
        \fill [white!80!black] (2.5cm,-1ex) rectangle (\textwidth+2.1cm,2.5ex);
        \node [
        fill=gray,
        anchor= base east,
        rounded rectangle,
        minimum height=3.5ex] at (3cm,0) {
            \textbf{\thesection}
        };
    }%
}
\titleformat{\section}{\huge}{\titlebar}{0.1cm}{}

%\newcommand\titlebarsss{%
%    \tikz[baseline,trim right=3cm] {
%        \fill [white!80!black] (2.5cm,-1ex) rectangle (\textwidth+2.6cm,2.5ex);
%        {
%            \textbf{\thesubsection}
%        };
%    }%
%}
%\titleformat{name=\subsection,numberless}{\large}{\titlebarsss}{0.1cm}{}

\makeatletter
\newcommand\titlebar@@{%
\tikz[baseline,trim left=3.1cm,trim right=3cm] {
    \fill [white!80!black] (2.5cm,-1ex) rectangle (\textwidth+3.1cm,2.5ex);
}}
\newcommand\titlebar@{%
\tikz[baseline,trim left=2.1cm,trim right=3cm] {
    \fill [white!80!black] (2.5cm,-1ex) rectangle (\textwidth+2.3cm,2.5ex);
    \node [
        fill=gray,
        anchor= base east,
        rounded rectangle,
        minimum height=3.5ex] at (3cm,0) {
        \textbf{\thesubsection}
    };
}}
\newcommand\titlebars{\@ifstar\titlebar@@\titlebar@}
\titleformat{\subsection}{\large}{\titlebars}{0.1cm}{}

\makeatletter
\newcommand\titlebars@@{%
\tikz[baseline,trim left=3.1cm,trim right=3cm] {
    \fill [white!80!black] 2.5cm,-1ex) rectangle (\textwidth+3.1cm,2.5ex);
}}
\newcommand\titlebars@{%
\tikz[baseline,trim left=1.6cm,trim right=3cm] {
    \fill [white!80!black] (2.5cm,-1ex) rectangle (\textwidth+1.8cm,2.5ex);
    \node [
        fill=gray,
        anchor= base east,
        rounded rectangle,
        minimum height=3.5ex] at (3cm,0) {
        \textbf{\thesubsubsection}
    };
}}
\newcommand\titlebarss{\@ifstar\titlebars@@\titlebars@}
\titleformat{\subsubsection}{\large}{\titlebarss}{0.1cm}{}

\usepackage{geometry}

\usepackage{titletoc}

\titlecontents{section}[2.3em]
  {}
  {\bfseries\contentslabel[\thecontentslabel.0]{2em}\MakeUppercase}
  {\hspace*{-2.3em}\bfseries\MakeUppercase}
  {\titlerule*[1pc]{.}\contentspage}
\titlecontents{subsection}[4.6em]
  {}
  {\bfseries\contentslabel{2em}}
  {\hspace*{-2.3em}\bfseries}
  {\titlerule*[1pc]{.}\contentspage}
\titlecontents{subsubsection}[6.9em]
  {}
  {\bfseries\contentslabel{2em}\itshape\space}
  {\hspace*{-2.3em}\bfseries}
  {\titlerule*[1pc]{.}\contentspage}

\makeatletter
\renewcommand\tableofcontents{%
  \section*{\centerline{\MakeUppercase{\contentsname}}
    \@mkboth
      {\MakeUppercase\contentsname}
      {\MakeUppercase\contentsname}
  }%
  \@starttoc{toc}%
}
\makeatother

\newcommand{\example}[3]{\begin{tcolorbox}[title=\large \examplecounter \hfill\small\textbf{"#1"}]
#2
\begin{tcolorbox}[title=\small \textbf{RJEŠENJE},colback=white]
\begin{center}
#3
\hfill $\vartriangle$
\end{center}
\end{tcolorbox}
\end{tcolorbox}}

\begin{document}

\begin{titlepage}
\raggedright
\rule{1pt}{\textheight}
\hspace{0.05\textwidth}
\parbox[b]{0.75\textwidth}{

{\large{Prirodno-matematički fakultet\\
Univerzitet u Tuzli\\
Odsjek: \textbf{Matematika}\\
Predmet: \textbf{Uvod u programske pakete}}}\\[7\baselineskip]
{\Huge\bfseries Skupovi}\\[1\baselineskip]
{\large\textit{-seminarski rad-}}\\[5\baselineskip]
{\Large\textsc{halilović adis}}

\vspace{0.5\textheight}

{\noindent Oktobar 2019}\\[\baselineskip]
}
\end{titlepage}

\newgeometry{right=2cm,top=5cm}
\hoffset -0.5cm
\thispagestyle{empty}
\tableofcontents
\clearpage

\addcontentsline{toc}{section}{sadržaj}
\clearpage

\addtocontents{toc}{\vspace{1em}}
\clearpage

\addcontentsline{toc}{section}{Tables}
\addtocontents{toc}{\vspace{1em}}
\clearpage

\pagenumbering{arabic}
\restoregeometry

\newpage

\newgeometry{bottom=1in, top=1in}
\section{Skupovi}

\bigskip\bigskip
\begin{Large}
Jedan od osnovnih ljudskih nagona jeste da sortira i klasificira stvari. Razmislite, naprimjer. Koliko različitih skupova ste vi član? Ako krenete sa nekim jednostavnim kategorijama, kao jeste li muško ili žensko, vaša starosna grupa, država u kojoj živite. Tada ćete možda razmišljati i o etničkoj pripadnosti vaše porodice, sociološko-ekonomskoj grupi, nacionalnosti. Ovo su samo neke od mnogobrojnih načina kako bi mogli opisati sebe drugim ljudima.

Kakva je korist od ovakve kategorizacije? Kao što ćete vidjeti u nastavku ovog poglavlja, sortirajući stvari po skupovima pomaže vam organizirati i urediti vaš svijet. U mogućnosti ste da vladate velikim količinama informacija. Skupovi su učevni alat koji pomaže da se odgovori na pitanje: "Koje su karakteristike grupe?"

Skupovi postavljaju temelje ostalim matematičkim oblastima, poput logike i apstraktne algebre. Ustvari, knjiga \textit{El\' ements de Math\' ematique}, napisana od strane grupe francuskog matematičara pod pseudonimom "Nicolas Bourbaki", kaže: "U današnjici je moguće, logički govoreći: "U današnjici je moguće, logički rečeno, "
\bigskip
\begin{tcolorbox}
Skupovi su osnovni alat za učenje pa čak i za malu djecu. Kao bebe, oni nauče razlikovati "mene" od "mama" i "tata". Kao mališani nauče razlikovati i kategorisati objekte kao članove skupa po veličini, boji, ili obliku. TV emisija "Ulice sezama" uči djecu da prave skupove kroz igru. "Jedna od tih stvari je različita od ostalih."
\end{tcolorbox}
\end{Large}
\newpage

\subsection{Koncepti skupa}

Svakodnevno srešemo skupove u našim životima u raznim situacijama. Skup je kolekcija objekata, koji se nazivaju \textbf{elementi} ili \textbf{članovi} skupa. Naprimjer, Sjedinjena Američka država je kolekcija ili skup 50 manjih država. 50 pojedinih država su članovi ili elementi jednog skupa koji se naziva Sjedinjene Američke države.
Skup je \textbf{dobro definisan} ako se njegov sadržaj može jasno odrediti. Skup trenutnih sudija koji služe SAD vrhovnom sudu je dobro definisan skup jer njihovi članovi i sudije mogu biti imenovani. Skup tri najbolja auta nije dobro definisan skup jer riječ \textit{najbolji} različiti ljudi različito tumače. U ovom tekstu, mi koristimo samo dobro definisane skupove.

Za označavanje skupa obično se koriste sljedeće tri metode:
\begin{center}
opis\\
popis ili obrazac\\
notacija skupa
\end{center}
Način označavanja skupa \textbf{opisom} prikazan je u primjeru 1.
\example{Opis skupova}{Napisati opis skupa koji sadrži elemente ponedjeljak, utorak, srijeda, četvrtak, petak, subota, nedjelja.}{Naziv skupa je "Dani u sedmici"}

Smještanje elemenata skupa unutar velikih zagrada "$\{ \}$", naziva se ruster forma. Zagrade su bitan dio notacije jer ona označavaju članove skupa.\\
Naprimjer, $\{1, 2, 3\}$ predstavlja notaciju za skup kojem pripadaju elemnti 1, 2 i 3, ali $(1, 2, 3)$ i $[1, 2, 3]$ nisu skup iz razloga što velike i srednje zagrade n prikazuju skup.

Skupo se inače imenuju sa velikim štampanim slovima. Naprimjer, ime koje se obično bira za skup prirodnih brojeva je \textbf{N}.\\
\begin{tcolorbox}
\begin{center}
\textbf{\textsc{PRIRODNI BROJEVI}}\\
$N = \{1, 2, 3, 4, 5, ... \}$
\end{center}
\end{tcolorbox}
Tri tačke nakon 5, nazivaju se \textit{elipsa}, i označavaju da elementi skupa nastavljaju dalje na isti način. Ako se nakon elipse pojavi neki elemenat, znači da elementi idu na isti način sve do tog elementa uključujući i njega. Ovakva notacija prikazana je na primjeru 2b.

\begin{tcolorbox}[title=\large \examplecounter\hfill\small\textbf{"Skup u formi popisa"}]
Predstaviti sljedeće u formi popisa.\begin{enumerate}[label=\alph*),leftmargin=0.5cm]
\item Skup $A$ je skup prirodnih brojeva manjih od 4.
\item Skup $B$ je skup prirodnih brojeva manjih ili jednako 70.
\item Skup $P$ je skup planeta sunčevog sistema.
\end{enumerate}
\begin{tcolorbox}[title=\small \textbf{RJEŠENJE},colback=white]
\begin{center}
\begin{enumerate}[label=\alph*),leftmargin=0.5cm]
\item Prirodni brojevi manji od 4 su 1, 2 i 3. Prema tome, skup $A$ u formi popisa je $A=\{1,2,3\}$
\item $B=\{1, 2, 3, 4, ... , 70\}$. Broj 70 nakon 3 tačke znači da se brojevi nastavljaju na isti način sve do 70.
\item $P=\{$Merkur, Venera, Zemlja, Mars, Jupiter, Saturn, Uran, Neptun, Pluton$\}$.
\end{enumerate}
\end{center}
\end{tcolorbox}
\end{tcolorbox}

\begin{tcolorbox}[title=\large \examplecounter \hfill\small\textbf{"Riječ \textit{uključujući}"}]
Sljedeće napiši u formi popisa.
\begin{enumerate}[label=\alph*),leftmargin=0.5cm]
\item Skup prirodnih brojeva između 5 i 8.
\item Skup prirodnih brojeva između 5 i 8, uključujući.
\end{enumerate}
\begin{tcolorbox}[title=\small \textbf{RJEŠENJE},colback=white]
\begin{center}
\begin{enumerate}[label=\alph*),leftmargin=0.5cm]
\item $A=\{6, 7\}$.
\item $B=\{5, 6, 7, 8\}$. Primjetite da riječ \textit{uključujuči} pokazuje na to da su i brojevi između 5 i 8 unutar skupa.
\end{enumerate}
\end{center}
\end{tcolorbox}
\end{tcolorbox}

Simbol \textbf{$\in$}, koji se čita, elemenat je, koristi se za označavanje članova u skupu. U primjeru 3, 6 je elemenat skupa A, pišemo $6\in A$. Također možemo napisati $6\in \{6, 7\}$. Isto tako $8\notin A$, što znači da 8 nije elemenat skupa A.

Notacija za kreiranje skupa (koja se ponekad naziva i notacija generatra skupa) može se koristiti da simbolizira skup. Navedena notacija skupa se često koristi u algebri. U nastavku primjer ilustrira njegov oblik.\\
\vfill
\begin{pspicture}(0,0)(11,2.5)
%\psgrid(0,0)(11,2.5)
\rput(1,2){$D$}
\rput(2,1.967){$\textbf{=}$}
\rput(3,2){$\{$}
\rput(5,2){$\textbf{x}$}
\rput(7,2){$\textbf{|}$}
\rput(10,2){Uslov(i) $\}$}
\rput(1,1.5){$\uparrow$}
\rput(2,1.5){$\uparrow$}
\rput(3,1.5){$\uparrow$}
\rput(5,1.5){$\uparrow$}
\rput(7,1.5){$\uparrow$}
\rput(10,1.5){$\uparrow$}

\rput(1,0.7){Skup $D$}
\rput(2,0.7){je}
\rput(3,0.7){skup}
\rput(5,0.7){svih}
\rput(5,0.4){elemenata}
\rput(5,0.1){$x$}
\rput(7,0.7){takvih}
\rput(7,0.4){da}
\rput(10,0.7){$x$ zadovoljava uslove}
\rput(10,0.4){da bi bio dio skupa}
\end{pspicture}
\medskip

Uzmimo u obzir $E=\{x|x\in N$ i $x\>10\}$. Izraz se čita kao "Skup $E$ je skup svih elemenata x takvih da je x veće od 10." Uslovi koje $x$ mora ispuniti da bi bio elemenat skupa je $x\in N$, što znači da $x$ mora biti prirodan broj, i $x\> 10$, što znači da $x$ mora biti veće od 10. Brojevi koji ispunjavaju oba uvjeta su elementi skupa prirodnih brojeva većih od 10. Skup u formi popisa izgleda ovako
$$E=\{11, 12, 13, 14, ...\ \}$$

\begin{tcolorbox}[title=\large \examplecounter \hfill\small\textbf{"Korištenje notacije za kreiranje skupova"}]
\begin{enumerate}[label=\alph*),leftmargin=0.5cm]
\item Napisati skup $B=\{1, 2, 3, 4, 5\}$ u obliku notacije za kreiranje skupa
\item Napiši, riječima, kako bi pročitao notaciju za kreiranje skupa $B$
\end{enumerate}
\begin{tcolorbox}[title=\small \textbf{RJEŠENJE},colback=white]
\begin{center}
\begin{enumerate}[label=\alph*),leftmargin=0.5cm]
\item Prema tome da se skup $B$ sastoji od prirodnih brojeva manjih od 6, pišemo $$B=\{x|x\in N\text { i } x< 6\}$$ Još jedan prihvatljiv odgovor je $B=\{x|x\in N$ i $x\leqslant 5\}$.
\item Skup $B$ je skup svih elemenata $x$ takvih da je $x$ prirodan broj i da je manji od 6.
\end{enumerate}
\end{center}
\end{tcolorbox}
\end{tcolorbox}

\begin{tcolorbox}[title=\large \examplecounter \hfill\small\textbf{"Forma popisa u notaciji građenja skupova"}]
\begin{enumerate}[label=\alph*),leftmargin=0.5cm]
\item Napisati skup $S = \{$Maine, Maryland, Massachusetts, Michigen, Minnesota, Mississippi, Missouri, Montana$\}$ u notaciji građenja skupova.
\item Napisati riječima kako bi pročitao/la skup $S$ u notaciji iz a).
\end{enumerate}
\begin{tcolorbox}[title=\small \textbf{RJEŠENJE},colback=white]
\begin{center}
\begin{enumerate}[label=\alph*),leftmargin=0.5cm]
\item $S=\{x|x$ je država u SAD-u čije ime počinje sa M$\}$.
\item Skup $S$ je skup svih elemenata x takvih da je x država SAD-a čije ime počinje sa slovom M.
\end{enumerate}
\end{center}
\end{tcolorbox}
\end{tcolorbox}

\begin{tcolorbox}[title=\large \examplecounter \hfill\small\textbf{"Notacija građenja skupova u formi popisa"}]
Napisati $A=\{x|x\in N$ i $2\leqslant x<8\}$ u formi popisa.
\begin{tcolorbox}[title=\small \textbf{RJEŠENJE},colback=white]
\begin{center}
$A=\{2, 3, 4, 5, 6, 7\}$
\end{center}
\end{tcolorbox}
\end{tcolorbox}

Za skup se kaže da je \textbf{konačan}, ako ne sadrži elemente ili broj elemenata u skupu je prirodan broj. Skup $B=\{2, 4, 6, 8, 10\}$ je konačan skup zato što je broj elemenata skupa 5, a 5 je prirodan broj. Skup koji nije konačan naziva se \textbf{beskonačan} skup. Skup brojivih brojeva je jedan primjer beskonačnog skupa. O beskonačnim skupovima govorit ćemo u Poglavlju ...\par\smallskip
Još jedan bitan koncept jeste jednakost skupova.\smallskip
\begin{tcolorbox}
Skup $A$ \textbf{jendak} je skupu $B$, simbolično napisano $A=B$, ako i samo ako skup $A$ sadrži tačno iste elemente kao skup $B$.
\end{tcolorbox}\smallskip
\noindent Naprimjer, ako skup $A=\{1, 2, 3\}$ i skup $B=\{3, 1, 2\}$, onda $A=B$ zato što oba sadrže tačno iste elemente. Radoslijed elemenata u skupu nije bitan. Ako su dva skupa jednaka onda moraju oba biti sačinjnena od istog broja elemenata. Broj elemenata skupa naziva se \textit{osnovnu/kardinalni broj}.\smallskip
\begin{tcolorbox}
\textbf{Osnovni/Kardinanli broj} skupa $A$, simbolično napisanon$n(A)$, je broj elemenata skupa $A$.
\end{tcolorbox}\smallskip
Oba skupa $A=\{1, 2, 3\}$ i $B=\{$Engleska, Francuska, Japan$\}$ imaju osnovni/kardinalni broj 3; to jest, $n(A)=3$, i $n(B)=3$. Kažemo da skup $A$ i skup $B$ oba imaju kardinalnost 3.\par
Za dva skupa se kaže da su \textbf{ekvivalentna} ako oba posjeduju isti broj elemenata.\smallskip
\begin{tcolorbox}
Skup $A$ je \textbf{ekvivalentan} skupu $B$ ako i samo ako $n(A)=n(B)$.
\end{tcolorbox}\smallskip
\noindent Bilo koji skupovi koju su jednaki moraju biti i ekvivalenti. Međutim, nisu svi ekvivalentni skupovi jednaki. Skupovi $D=\{a, b, c\}$ i $E=\{$apple, orange, pear$\}$ su ekvivalenti, stoga što oba imaju isti osnovni/kardinalni broj, 3. Zbog toga što se elementi ralikuju, međutim, skupovi nisu jednaki.\par
Dva skupa koja su ekvivalentna i  imaju istu kardinalnost mogu se dopisati 1-1. Skup $A$ i skup $B$ mogu se dopisati 1-1 ako svaki elemenat skupa $A$ može biti povezan sa tačno jednim elementom skupa $B$ i svaki elemenat skupa $B$ može biti povezan sa tačno jednim elementom skupa $A$. Naprimjer, dopis 1-1 postoji između imena studenata sa liste razreda i sa njihovim identifikacionim brojevima zato što povezati njihova imena sa njihovim brojevima.\smallskip
\begin{tcolorbox}
Uzmimo u obzir skup $B$, ime brenda produkta, i skup $D$, pića.
\begin{center}
$B=\{$Meggle, Biljana, Oaza, Zlatna džezva$\}$\\
$D=\{$čaj, mlijeko, kafa, voda$\}$
\end{center}
Dva zarličita dopisa 1-1 za skupove $B$ i $D$ slijede.\medskip
\begin{tcolorbox}[colback=white]
\smallskip
\begin{center}
\begin{pspicture}(0,0)(7,3)
%\psgrid(0,0)(7,3)
\rput(3.5,0){$D=\{$čaj, mlijeko, kafa, voda$\}$}
\rput(3.5,1){$B=\{$Meggle, Biljana, Oaza, Zlatna džezva$\}$}
\rput(3.5,2){$D=\{$čaj, mlijeko, kafa, voda$\}$}
\rput(3.5,3){$B=\{$Meggle, Biljana, Oaza, Zlatna džezva$\}$}

\psline[arrows=->](1.6,2.8)(2.25,2.2)%megle,čaj
\psline[arrows=->](2.85,2.8)(3.25,2.2)%biljana,mlijeko
\psline[arrows=->](4.05,2.8)(4.35,2.2)%oaza, kafa
\psline[arrows=->](5.7,2.8)(5.3,2.2)%z.dž, voda

\psline[arrows=->](1.6,0.8)(3.25,0.2)%done
\psline[arrows=->](2.85,0.8)(2.25,0.2)%done
\psline[arrows=->](4.05,0.8)(5.3,0.2)
\psline[arrows=->](5.7,0.8)(4.35,0.2)
\end{pspicture}
\smallskip
\end{center}
\end{tcolorbox}
\end{tcolorbox}
Drugi dopisi 1-1 su mogući između skupova $B$ i $D$. Da li znate koje piće ide sa kojim imenom brenda produkta?

\subsubsection{\textsc{nula ili prazan skup}}
Neki skupovi ne sadrže ni jedan elemenat, kao naprimjer skup zebri koje su u ovoj sobi.
\begin{tcolorbox}
Skup koji ne sadrži elemente naziva se \textbf{prazan skup} ili \textbf{nula skup} i simbolično se označava sa $\{\ \}$ ili $\varnothing$.
\end{tcolorbox}
Imajte na umu da ${\varnothing}$ nije prazan skup. Ovaj skup sadrži elemenat $\varnothing$ i ima kardinalnost 1. Skup $\{0\}$ također nije prazan skup zato što sadrži elemenat $0$. I on ima kardinalnost 1.
\begin{tcolorbox}[title=\large \examplecounter \hfill\small\textbf{"Rješenja prirodnih brojeva"}]
Naznalite skup prirodnih brojeva koji zadovoljavaju jednačinu $x+2=0$.
\begin{tcolorbox}[title=\small \textbf{RJEŠENJE},colback=white]
\begin{center}
Vrijednosti koje zadovoljavaju jednačinu moraju biti takve da jednačina bude tačna. Samo broj $-2$ zadovoljava ovu jednačinu. Zbog toga što $-2$ nije prirodan bro, rješenje iove jednačine je $\{\ \}$ ili $\varnothing$.
\end{center}
\end{tcolorbox}
\end{tcolorbox}

\subsubsection{\textsc{univerzalni skup}}
Još jedan bitan skup jeste \textbf{univerzalni skup}.
\begin{tcolorbox}
\textbf{univerzalni skup}, simbolično napisan sa $U$, je skup koji sadrži sve elemente za bilo koju diskusiju.
\end{tcolorbox}
Kada je dat univerzalni skup, samo elementi univerzalnog skupa mogu se razmatrati prilikom rješavanja problema. Ako je, naprimjer, univerzalni skup za određeni problem definisan kao $U=\{1, 2, 3, 4, .., 10\}$, onda samo prirodni brojevi od 1 do 10 mogu biti korišteni tokom problema.

\begin{tcolorbox}[title=\textbf{ZADACI ZA VJEŽBU}]
\addcontentsline{toc}{subsection}{\protect\numberline{}\textsc{zadaci za vježbu}}
\begin{minipage}{0.5\textwidth}
\textbf{KONCEPT/VJEŽBE PISANJA}\\
\textit{U zadacima od 1 do 8, odgovoriti na pitanja punom rečenicom.}
\begin{enumerate}
\item Šta je skup?
\item Koje su tri varijante kojima skup može biti zapisan? navedite primjer za svaku od njih.
\item Šta je beskonačan skup?
\item Šta je konačan skup?
\item Šta su jednaki skupovi?
\item Šta su ekvivalenti skupovi?
\item Šta je kardinalni broj skupa?
\item Šta je univerzalni skup?
\item Šta je prazan skup?
\end{enumerate}
\end{minipage}
\begin{minipage}{0.5\textwidth}
\textbf{ZADACI ZA VJEŽBU VJEŠTINE}\\
\textit{U zadacima od 10 do 12, odredite da li su skupovi dobro definisani.}
\begin{enumerate}
\setcounter{enumi}{9}
\item Skup najboljih internet sajtova
\item Skup ljudi koji posjeduju velike pse
\item Skup učenika razreda koji su rođeni u Bosni i Hercegovini
\end{enumerate}
\textit{U zadacima od 13 do 24, odredite da li je skup konačan ili beskonačan}.
\begin{enumerate}
\setcounter{enumi}{12}
\item $\{1, 3, 5, 7, ...\}$
\item Skup parnih brojeva većih od 15
\item Skup neparnih brojeva većih od 15
\item Skup vrabaca koji cvrkuću u parku 4. jula u 10 sati.
\end{enumerate}
\end{minipage}
\end{tcolorbox}


\end{document}